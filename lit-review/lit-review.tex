% Created 2012-10-07 Sun 17:09
\documentclass[11pt]{article}
\usepackage[latin1]{inputenc}
\usepackage{fixltx2e}
\usepackage{url}
\usepackage{graphicx}
\usepackage{minted}
\usepackage{color}
\usepackage{longtable}
\usepackage{float}
\usepackage{wrapfig}
\usepackage{soul}
\usepackage{textcomp}
\usepackage{amsmath}
\usepackage{marvosym}
\usepackage{wasysym}
\usepackage{latexsym}
\usepackage{amssymb}
\usepackage[linktocpage,
  pdfstartview=FitH,
  colorlinks,
  linkcolor=blue,
  anchorcolor=blue,
  citecolor=blue,
  filecolor=blue,
  menucolor=blue,
  urlcolor=blue]{hyperref}
\usepackage{attachfile}
\tolerance=1000
\usepackage{amsmath}
\usepackage[super,sort&compress]{natbib}
\usepackage{natmove}
\usepackage{underscore}
\usepackage{makeidx}
\usepackage[section]{placeins}
\usepackage{xcolor}
\usepackage{adjustbox}
\usepackage{url}
\usepackage{anysize}
\marginsize{1in}{1in}{1in}{1in}
\makeindex
\providecommand{\alert}[1]{\textbf{#1}}

\title{Literature Review: Using DFT to study active surfaces in the oxygen eveolution reaction (OER)}
\author{Zhongnan Xu}
\date{10/08/12 Monday}
\hypersetup{
  pdfkeywords={},
  pdfsubject={},
  pdfcreator={Emacs Org-mode version 7.8.11}}

\begin{document}

\maketitle



\section*{Problem}
\label{sec-1}

The specific problem addressed is how to accurately model surface
coverages of oxides in water with an applied potential and varying
pH. 
This problem is relevant because of a recently developed atomistic 
thermodynamic framework for studying the oxygen evolution reaction
(OER).
This framework allows us to calculate theoretical activities of oxides in 
OER off of the adsorption energies of *O, *OH, and *OOH. 
However, initial studies of these mechanisms were done on ideal surfaces, and
coverages were not made to be consistent with actual OER conditions.

Recently, two papers from separate research groups have attempted to
build on this theory to more accurately mimic actual surfaces of
oxides in water with an applied potential and varying pH. 
The two papers are titled ``Identifying active surface phases for metal oxide
electrocatalysts: a study of manganese oxide bi-functional catalysts
for oxygen reduction and water oxidation catalysis'' and ``Water
Oxidation on Pure and Doped Hematite (0001) Surfaces: Prediction of Co
and Ni as Effective Dopants for Electrocatalysis'', and their
corresponding authors are Jan Rossmeisl and Emily A. Carter,
respectively.
Detailed references can be seen in the footnotes and the full article
can be found in the folder with this assignment.\footnote{Su, H.; Gorlin, Y.; Man, I. C.; Calle-vallejo, F.; Norskov, J. K.;
Jaramillo, F.; Rossmeisl, J. Physical Chemistry Chemical Physics 2012,
14, 14010
 }\textsuperscript{,}\,\footnote{Liao, P.; Keith, J. A; Carter, E. A Journal of the American Chemical
Society 2012, 134, 13296
 }
\section*{Use of DFT}
\label{sec-2}

Both article's overall goal was to evaluate catalyst's performance in
the oxygen evolution reaction (OER) using the same atomistic framework
constructed in a previous article \footnote{Man, I.; Su, H.; Calle-Vallejo, F. ChemCatChem 2011, 3, 1159
 }.
By using density functional theory (DFT) as a tool to model oxide
surfaces and calculate adsorption energies of water electrolysis
intermediates, one could then use these adsorption energies to
calculate reactivities of oxide surfaces that were in qualitative
agreement with experiments.
This work helped validate the approach with idealized surfaces, so the
next question is what the actual surfaces were in OER conditions,
which includes effects of water solvation, applied potential, and pH
effects.

Rossmeisl and coworkers took an approach that involved taking a bulk
Pourbaix diagram and `upgrading' it to include surface stability with
respect to pH and applied potential.
Their main equation of modeling dissociation and adsorption of water
onto idealized surfaces is shown below.

\begin{equation}\begin{split}
X^\ast &+ (N_{\mathrm{O^\ast}} + N_{\mathrm{HO^\ast}})\mathrm{H_2O}(l)
\rightarrow \\
& (N_{\mathrm{O^\ast}} + N_{\mathrm{HO^\ast}} +
N^\ast)_{\mathrm{ads}} + (2N_{\mathrm{O^\ast}} +
N_{\mathrm{HO^\ast}})\mathrm{H^+} + (2N_{\mathrm{O^\ast}} + N_{\mathrm{HO^\ast}})\mathrm{e^-}
\end{split}\end{equation}

This reaction describes the interaction of water with
surfaces. Starting with a clean surface with a total of $X^\ast$
sites, water will adsorb and produce protons, electrons, and adsorbed
species of either $\mathrm{O^{\ast}}$ or $\mathrm{HO^{\ast}}$. In producing these
intermediates, there will be a release of an electron and proton, and
higher pHs and higher applied potentials will stabilize these
intermediates. The Gibbs free energy of a surface under these
conditions is shown below.

\begin{equation}
\begin{split}
G_{\mathrm{surf}} &= E_{(N_{\mathrm{O^\ast}} + N_{\mathrm{HO^\ast}} + N_\ast)_{\mathrm{ads}}} - E_{\mathrm{X^{\ast}}} -
(N_{\mathrm{O^\ast}} + N_{\mathrm{HO^\ast}}) E_{\mathrm{H_2O(g)}} \\ 
&\quad + \dfrac{2N_{\mathrm{O^\ast}} +
N_{\mathrm{HO^\ast}}}{2} E_{\mathrm{H_2(g)}} + \mathrm{\Delta ZPE -
T\Delta S} \\
&\quad - (2N_{\mathrm{O^\ast}} + N_{\mathrm{HO^\ast}})(eU +
k_\mathrm{B}T\mathrm{ln}10\mathrm{pH})
\end{split}
\end{equation}

\end{document}