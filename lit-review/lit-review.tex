% Created 2012-10-07 Sun 21:50
\documentclass[11pt]{article}
\usepackage[latin1]{inputenc}
\usepackage{fixltx2e}
\usepackage{url}
\usepackage{graphicx}
\usepackage{minted}
\usepackage{color}
\usepackage{longtable}
\usepackage{float}
\usepackage{wrapfig}
\usepackage{soul}
\usepackage{textcomp}
\usepackage{amsmath}
\usepackage{marvosym}
\usepackage{wasysym}
\usepackage{latexsym}
\usepackage{amssymb}
\usepackage[linktocpage,
  pdfstartview=FitH,
  colorlinks,
  linkcolor=blue,
  anchorcolor=blue,
  citecolor=blue,
  filecolor=blue,
  menucolor=blue,
  urlcolor=blue]{hyperref}
\usepackage{attachfile}
\tolerance=1000
\usepackage{anysize}
\marginsize{1in}{1in}{1in}{1in}
\providecommand{\alert}[1]{\textbf{#1}}

\title{Literature Review: Using DFT to study active surfaces in the oxygen evolution reaction (OER)}
\author{Zhongnan Xu}
\date{10/08/12 Monday}
\hypersetup{
  pdfkeywords={},
  pdfsubject={},
  pdfcreator={Emacs Org-mode version 7.8.11}}

\begin{document}

\maketitle



\section{Introduction}
\label{sec-1}

  The specific problem addressed is how to accurately represent active
  surfaces of oxides in the oxygen evolution reaction (OER).
  This problem is relevant because of a recently developed atomistic 
  thermodynamic framework for studying OER.
  This framework allows us to calculate theoretical activities of oxides in 
  OER off of the adsorption energies of *O, *OH, and *OOH. 
  Initial studies of these mechanisms were done on ideal surfaces with
  an arbitrary coverage of intermediates, and water was not assumed to
  naturally dissociate onto these surfaces.
  
  Recently, two papers from separate research groups have attempted to
  build on this theory to more accurately mimic actual surfaces of
  oxides in water.
  The two papers are titled ``Identifying active surface phases for metal oxide
  electrocatalysts: a study of manganese oxide bi-functional catalysts
  for oxygen reduction and water oxidation catalysis'' and ``Water
  Oxidation on Pure and Doped Hematite (0001) Surfaces: Prediction of Co
  and Ni as Effective Dopants for Electrocatalysis'', and their
  corresponding authors are Jan Rossmeisl and Emily A. Carter,
  respectively.
  Detailed references can be seen in the footnotes and the full article
  can be found in the folder with this assignment.\footnote{Su, H.; Gorlin, Y.; Man, I. C.; Calle-vallejo, F.; Norskov, J. K.;
Jaramillo, F.; Rossmeisl, J. Physical Chemistry Chemical Physics 2012,
14, 14010
 }\textsuperscript{,}\,\footnote{Liao, P.; Keith, J. A; Carter, E. A Journal of the American Chemical
Society 2012, 134, 13296
 }
  
\section{Methods}
\label{sec-2}

  Both article's overall goal was to evaluate catalyst's performance in
  the oxygen evolution reaction (OER) using the same atomistic framework
  constructed in a previous article \footnote{Man, I.; Su, H.; Calle-Vallejo, F. ChemCatChem 2011, 3, 1159
 }.
  However, Rossmeisl's goal was to model both the thermodynamics and
  kinetics of OER and construct a theoretical current-voltage curve,
  while Carter's goal was to compare the thermodynamics of pure and
  doped hematite. 
  By using density functional theory (DFT) as a tool to model oxide
  surfaces and calculate adsorption energies of water electrolysis
  intermediates, one could then use these adsorption energies to
  calculate reactivities of oxide surfaces that were in qualitative
  agreement with experiments.
  This work helped validate the approach with idealized surfaces, so the
  next question is what the actual surfaces were in OER conditions,
  which includes effects of water solvation, applied potential, and pH
  effects.
  
\subsection{Rossmeisl Approach: Effects of pH and electrode potential}
\label{sec-2-1}

   Rossmeisl et al. (2012) took an approach that involved taking a bulk
   Pourbaix diagram and `upgrading' it to include surface stability with
   respect to pH and applied potential.
   They studied the MnO$_{x}$ system in OER conditions, which include
   Mn$_{3}$O$_{4}$, Mn$_{2}$O$_{3}$, and MnO$_{2}$.
   Their goal was to use DFT to find both thermodynamic and kinetic data
   and model a fully self-consistent current-voltage curve of a MnO$_{x}$
   electrode and compare them to experiments.
   In order for this curve to be self-consistent, they had to take into
   account changes of both bulk \textbf{and} surface phases with respect to
   applied potential and pH.
   Their main equation of modeling dissociation and adsorption of water
   onto each of the idealized surfaces is shown below.
   
   \begin{equation}
   \begin{split}
   X^\ast &+ (N_{\mathrm{O^\ast}} + N_{\mathrm{HO^\ast}})\mathrm{H_2O}(l)
   \rightarrow \\
   & (N_{\mathrm{O^\ast}} + N_{\mathrm{HO^\ast}} +
   N^\ast)_{\mathrm{ads}} + (2N_{\mathrm{O^\ast}} +
   N_{\mathrm{HO^\ast}})\mathrm{H^+} + (2N_{\mathrm{O^\ast}} + N_{\mathrm{HO^\ast}})\mathrm{e^-}
   \end{split}
   \end{equation}
   
   This reaction describes the interaction of water with
   surfaces. Starting with a clean surface with a total of $X^\ast$
   sites, water will adsorb and produce protons, electrons, and adsorbed
   species of either $\mathrm{O^{\ast}}$ or $\mathrm{HO^{\ast}}$. In producing these
   intermediates, there will be a release of an electron and proton, and
   higher pHs and higher applied potentials will stabilize these
   intermediates. The Gibbs free energy of a surface under these
   conditions is shown below.
   
   \begin{equation}
   \begin{split}
   G_{\mathrm{surf}} &= E_{(N_{\mathrm{O^\ast}} + N_{\mathrm{HO^\ast}} + N_\ast)_{\mathrm{ads}}} - E_{\mathrm{X^{\ast}}} -
   (N_{\mathrm{O^\ast}} + N_{\mathrm{HO^\ast}}) E_{\mathrm{H_2O(g)}} \\ 
   &\quad + \dfrac{2N_{\mathrm{O^\ast}} +
   N_{\mathrm{HO^\ast}}}{2} E_{\mathrm{H_2(g)}} + \mathrm{\Delta ZPE -
   T\Delta S} \\
   &\quad - (2N_{\mathrm{O^\ast}} + N_{\mathrm{HO^\ast}})(eU +
   k_\mathrm{B}T\mathrm{ln}10\mathrm{pH})
   \end{split}
   \end{equation}
   
   To use this equation, one models the same surface with different
   coverages of O and OH. Generally, the more adsorbates on the surface,
   the more free electrons and protons that need to be produced, and the
   more likely that surface will be present at higher potentials and
   pH's.
   
   Note that these surface Pourbaix diagrams are done in the context
   of a specific bulk Pourbaix diagram. From experiments, the bulk Pourbaix
   diagrams are assumed to be true, and then in each bulk region of the
   Pourbaix diagrams, the surface Pourbaix diagrams are constructed. They
   first did this for Mn$_{3}$O$_{4}$, Mn$_{2}$O$_{3}$, and MnO$_{2}$, and then
   made a general MnO$_{x}$ Pourbaix diagram that gave the most stable bulk
   surface phase. 
   
\subsection{Carter Approach: A detailed investigation on water dissociation}
\label{sec-2-2}

   Carter et al. (2012) studied hematite (Fe$_{2}$O$_{3}$) for photocatalytic water
   splitting.
   In order to attain a general idea of the surface, they did an
   extensive literature review of both experimental and computational
   investigations on the active surface facet and termination of hematite
   in wet conditions.
   They found that in water, the most likely surface is an
   H$_{3}$-O$_{3}$-Fe-Fe-R surface. 
   
   They opted to not include effects of pH and increasing electrode 
   potential, but did a far more extensive study on possible
   configurations of possible surface configurations.
   Starting with an ideal, oxygen terminated surface of hematite, 
   they asked the question, ``If we were to make surface with dangling 
   hydroxyl groups, which would be most stable? How many of these
   oxygens would be hydrated to make hydroxyl's, and what would the
   configuration of these bonds be?'' 
   To answer this question, they broke up their surface into a 7x7
   grid of a total of 49 points. 
   They then placed one hydrogen atom at each of these points at a height
   of 0.8 $\AA$ above the surface, allowed the hydrogen to
   relax.
   They would then calculate the adsorption energy as shown below.
   
   \begin{equation}
   \begin{split}
   G_{\mathrm{ads}} &= [E(\mathrm{O-terminated\;slab} + n\mathrm{H}) \\
   &\quad E(\mathrm{O-terminated\;slab}) - (n/2)E_{\mathrm{H_2}}]/n \\
   &\quad + \mathrm{\Delta ZPE} - T\Delta S
   \end{split}
   \end{equation}
   
   They then took the structure with the lowest energy and then did the
   same process again with a second hydrogen atom. 
   By doing this, they
   could be confident that they are getting the DFT predicted ground
   state of their system, and they could also systematically analyze
   increasing coverages of hydroxyl groups.
   Note that equation (3) is independent of how many hydroxyl groups are
   on, so it is possible to find the ground state structure.
   
   In addition to finding the most stable H$_{x}$-O-Fe-Fe-R surface, they
   also included affects of a monolayer of water.
   The orientations of the water molecules were chosen to maximize
   hydrogen bonding, and similar to the procedure for choosing the most
   stable hydroxylated surface, the most stable solvated surface, after
   relaxation, was chosen as well.
   
\section{Discussion}
\label{sec-3}

  When comparing and contrasting both techniques of accurately
  representing the surface phase in OER conditions, its important note
  the differing goals of both.
  Rossmeisl and coworkers were seeking
  theoretical current-voltage curves that could be directly comparable
  to experiments.
  The purpose of this was to use theoretical techniques to probe the
  fundamental characteristics of active phase
  as one increased the potential in a typical OER experiment.
  In contrast, Carter was seeking qualitative agreement between pure
  and doped hematite surfaces.
  Therefore, one could easily assume changes in the bulk and surface
  structures at differing pH's and voltages would not be drastically
  different across different dopings of cations.

  In assessing the reproducibility of the results, the piece of
  information that is most valuable in these surface calculations is
  the geometry of the adsorbates.
  Both articles give specific structural parameters of their surfaces,
  but only the Carter paper gives a detailed description of how the 
  hydroxylated surfaces were produced. 
  Rossmeisl does not describe the geometry of the HO* and HOO*
  adsorbates, and from personal experience, these geometries are 
  highly dependent on the initial conditions of the relaxations. 
  Therefore, it is unclear whether Rossmeisl's surfaces are in their
  ground state, and it is further impossible to reproduce their
  values.
  In contrast, Carter's paper gives a clear explanation on how the
  hydroxylated structures were constructed and geometries and total
  energies of the most stable structures.

\end{document}